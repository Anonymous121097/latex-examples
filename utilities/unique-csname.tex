\makeatletter

% combination of Henri Menke and Skillmon's answers to
%   https://tex.stackexchange.com/q/633706

\long\def\@uniquecsname@try#1{%
  \ifcsname #1\endcsname
    \expandafter\@firstoftwo
  \else
    \expandafter\@secondoftwo
  \fi
  % expand \@uniquecsname@generate twice
  {\expandafter\expandafter\expandafter\@uniquecsname@try
   \expandafter\expandafter\expandafter{\@uniquecsname@generate{#1}}}
  {#1}%
}

% if no random int primitive is provided, use the null char trick
\begingroup
\catcode`\^^@=12 % default catcode is 15 (invalid character)
\unexpanded{\endgroup % tokenize ^^@ with catcode 12
  \ifdefined\pdfuniformdeviate
    \def\@uniquecsname@generate#1{\number\pdfuniformdeviate\maxdimen}
  \else\ifdefined\uniformdeviate
    \def\@uniquecsname@generate#1{\number\uniformdeviate\maxdimen}
  \else
    \edef\@uniquecsname@generate#1{#1^^@}% the 0x00 null char
  \fi\fi
}

% start with empty csname (\csname\endcsname IS a valid cs)
\def\uniquecsname{\@uniquecsname@try{}}

% Q: Why no \uniquecmd that expands to an always-undefined control sequence?
% A: It's impossible since both \begingroup and \endgroup are unexpandable.
%    \undefine and \undefinecs will cover all the unexpandable use cases.

\long\def\undefine#1{% short form \undef is taken by etoolbox.sty
  \begingroup\expandafter\endgroup
  \expandafter\let\expandafter#1\csname\uniquecsname\endcsname}

\long\def\undefinecs#1{%
  \expandafter\undefine\csname #1\endcsname}
\makeatother
