\usepackage{hyperref}
\usepackage{xpatch}

\makeatletter
\ExplSyntaxOn

% list of label types (prefixes)
\clist_new:N \hy_@linkcolor_clist


%% new hyperref option "subcolorByLabelPrefix"
\newif\ifHy@subcolorByLabelPrefix
\Hy@subcolorByLabelPrefixfalse

\define@key{Hyp}{subcolorByLabelPrefix}[true]{%
  \Hy@boolkey{subcolorByLabelPrefix}{#1}%
}


%% new hyperref option "subcolor"
% syntax: subcolor={<label prefix>}{<color>}
\define@key{Hyp}{sublinkcolor}{
  \Hy@newsubcolor{link}#1
}

% example: \Hy@newsubcolor{link}{sec}
\def\Hy@newsubcolor#1#2{%
  \exp_args:Nc \Hy@@newsubcolor {@ #1 color@ #2} {#1} {#2}
}
% example: \Hy@newsubcolor \@linkcolor@sec{link}{sec}{blue}
\def\Hy@@newsubcolor#1#2#3#4{
  \HyColor@HyperrefColor {#4} #1
  \clist_put_left:cn {hy_@ #2 color_clist} {#3}
}
\ExplSyntaxOff


% originally defined by "hyperref.sty", used by \hyperref[]{}
% example: #1 = "\r@sec:key", #2 = "sec:key", #3 = "text"
\xpatchcmd\label@@hyperref
  {\hyper@@link}
  {\HyRef@testlabeltype#2::\\\hyper@@link}
  {}{\fail}

% originally defined by "hyperref.sty", used by \autoref and \autopageref
% local, inside a group openned by \HyRef@autoref
% example: #1 = "\hyper@@link", #2 = "sec:key"
\xpatchcmd\HyRef@autosetref
  {\HyRef@ShowKeysRef}
  {\HyRef@testlabeltype#2::\\\HyRef@ShowKeysRef}
  {}{\fail}

% originally defined by "nameref.sty", used by \ref, \pageref, and \nameref
% example: #1 = "\r@sec:key", #2 = "\@firstoffive", #3 = "sec:key"
\xpatchcmd\@setref
  {\ifx}
  {\HyRef@testlabeltype#3::\\\ifx}
  {}{\fail}

% store label type in \CurrentLabelPrefix
% example: \HyRef@testlabeltype sec:key::\\, \HyRef@testlabeltype key::\\
\def\HyRef@testlabeltype#1:#2:#3\\{%
  \ltx@ifempty{#2}{%
    \let\CurrentLabelPrefix\@empty
  }{%
    \ifHy@subcolorByLabelPrefix
      \ltx@IfUndefined{@linkcolor@#1}{}{%
        \def\CurrentLabelPrefix{#1}%
      }%
    \fi
  }%
}


% originally defined by "hycolor.sty", used by all ref commands
% example: #1 = "\@linkcolor"
\xpatchcmd\HyColor@UseColor
  {\expandafter}
  {\Hy@setsubcolor{#1}\expandafter}
  {}{\fail}

\ExplSyntaxOn
% example: \Hy@setsubcolor\@linkcolor
\def\Hy@setsubcolor#1{
  \clist_map_inline:cn { hy_ \cs_to_str:N #1 _clist }
    {
      \tl_set:Nn \l_tmpa_tl {##1}
      \tl_if_eq:NNT \CurrentLabelPrefix \l_tmpa_tl
        {
          \tl_set_eq:Nc #1 { \cs_to_str:N #1 @##1 }
          \clist_map_break:
        }
    }
}
\ExplSyntaxOff
\makeatother

\endinput


% why we must insert \HyRef@testlabeltype three different places
\hyperref[sec:key]{text}
  \label@hyperref[{sec:key}]{text}% <- patch here
    \label@@hyperref\r@sec:key {key}{text}
      \hyper@@link{\expandafter\@fifthoffive\r@sec:key}%
                  {\expandafter\@fourthoffive\r@sec:key \@empty\@empty}
                  {text}%
        % \hyper@link@[link]{...}{...}{...}
        %   \hyper@link{link}{section.1}{text}
            % \find@pdflink{link}{section.1}text\Hy@xspace@end\close@pdflink
            %   \pdfstartlink attr{/Border[0 0 0]/H/I/C[1 0 0]}goto name{section.1}
            %   \Hy@colorlink\@linkcolor
            %   text\Hy@xspace@end\close@pdflink


\autoref{sec:key}
  \HyRef@autoref\hyper@@link{sec:key}
    \HyRef@autosetref \r@sec:key {sec:key}{\hyper@@link}% <- patch here
      \hyper@@link{\expandafter\@fifthoffive\r@sec:key \@empty\@empty\@empty}
                  {\expandafter\@fourthoffive\r@sec:key \@empty\@empty\@empty}
                  {\HyRef@currentHtag
                  \expandafter\@firstoffive\r@sec:key \@empty\@empty\@empty
                  \null}
        % \hyper@link@[link]{...}{...}{...}
        %   \hyper@link{link}{section.1}{...}


\ref{sec:key}
  \T@ref{sec:key}
    \NR@setref{sec:key}\@firstoffive{sec:key}
      \NR@@setref\r@sec:key \@firstoffive {sec:key}
        \@setref\r@sec:key \@firstoffive{sec:key}% <- patch here
          \Hy@setref@link{1}{2}{title}{section.1}{}\@empty\@empty\@nil\@firstoffive
            \hyper@@link{}{secion.1}{1}
