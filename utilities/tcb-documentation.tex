\RequirePackage{tcolorbox}
\tcbuselibrary{documentation}

\ExplSyntaxOn
\makeatletter

%%
%% preparation
%%

% constant variable
\clist_const:Nn \c_tcb_doc_head_options_clist
  { name, parameter, description, label }

% local variables
\clist_new:N \l_tcb_doc_labelnames_clist
\clist_new:N \l_tcb_doc_cmdnames_clist
\clist_new:N \l_tcb_doc_full_heads_clist
\seq_new:N \l_tcb_doc_heads_seq

% temp variables
\clist_new:N \l_tcb_temp_clist
\prop_new:N \l_tcb_temp_prop
\seq_new:N \l_tcb_temp_seq
\tl_new:N \l_tcb_temp_tl

% variant form of expl3 function
\cs_generate_variant:Nn \clist_put_right:Nn {Nv}

% new tcb options
\tcbset{
  doc~ name/.store~ in=\kvtcb@doc@name,
  doc~ parameter/.store~ in=\kvtcb@doc@parameter,
  doc/para/.forward~ to=/tcb/doc~ parameter,
  doc/desc/.forward~ to=/tcb/doc~ description,
  doc/.unknown/.code = {
    \meta_pgfkeys_prefix_forward_to:nVV {/tcb/doc~}
      \pgfkeyscurrentname \pgfkeyscurrentvalue
  }
}

% init new options
\tcbset{doc~ name=, doc~ parameter=}

% #1 = path and key prefix
% #2 = key
% #3 = value
\cs_new:Nn \meta_pgfkeys_prefix_forward_to:nnn
  {
    \pgfkeysalso{#1 #2=#3}
  }
\cs_generate_variant:Nn \meta_pgfkeys_prefix_forward_to:nnn {nVV}


%%
%% {docCommands} env, poor syntax
%%

% #1 = tcb options
% #2 = clist of 3-tuple, "{name}{para}{desc}, {}{}{}, ..."
\NewDocumentEnvironment{docCommands}{ O{} m }
  {
    \tcbset{#1}
    \begin{tcb@manual@entry}
    \tcb_doc_init_per_env:
    \tcb_doc_parse_old_heads:n {#2}
    \nobreak\tcbset{before~ upper=}
    \kvtcb@doc@body@command@before
    \ignorespaces
  }
  {
    \ifvmode\else\unskip\fi
    \kvtcb@doc@body@command@after
    \end{tcb@manual@entry}
  }

\cs_new:Nn \tcb_doc_init_per_env:
  {
    \clist_map_inline:Nn \c_tcb_doc_head_options_clist
      { \tl_clear:c { kvtcb@doc@ ##1 } }
  }

% convert list of "{name}{para}{desc}" to list of "{name=, para=, desc=}",
% then call \tcb_doc_heads:V
% #1 = clist of itmes "{name}{para}{desc}"
\cs_new:Nn \tcb_doc_parse_old_heads:n
  {
    \clist_clear:N \l_tcb_temp_clist
    \clist_map_inline:nn {#1}
      { \tcb_doc_parse_old_tuple:nnn ##1 }
    \tcb_doc_heads:V { \l_tcb_temp_clist }
  }
\cs_new:Nn \tcb_doc_parse_old_tuple:nnn
  {
    \clist_put_right:Nn \l_tcb_temp_clist
      { {name=#1, para=#2, desc=#3, label=#1} }
  }

% #1 = clist of items "{name=..., para=..., desc=...}"
\cs_new:Nn \tcb_doc_heads:n
  {
    % init
    \clist_clear:N \l_tcb_doc_cmdnames_clist   % no-duplication label names
    \clist_clear:N \l_tcb_doc_labelnames_clist % no-duplication cmd names
    % call \tcb_doc_head_single:
    \seq_set_from_clist:Nn \l_tcb_doc_heads_seq {#1}
    \seq_pop_left:NN \l_tcb_doc_heads_seq \l_tcb_temp_tl
    \seq_if_empty:NTF \l_tcb_doc_heads_seq
      { % length == 1
        \tcb_doc_head_single:V \l_tcb_temp_tl
      }{ % length >= 2
        \tcb_doc_head_first:V \l_tcb_temp_tl
        \seq_pop_right:NN \l_tcb_doc_heads_seq \l_tcb_temp_tl
        \seq_if_empty:NF \l_tcb_doc_heads_seq
        { % length >= 3
          \seq_map_inline:Nn \l_tcb_doc_heads_seq
            { \tcb_doc_head_middle:n {##1} }
        }
        \tcb_doc_head_last:V \l_tcb_temp_tl
      }
  }
\cs_generate_variant:Nn \tcb_doc_heads:n {V}

% wrappers of \tcb_doc_head:nn
\cs_new:Nn \tcb_doc_head_single:n
  { \tcb_doc_head:nn {#1} {} }
\cs_new:Nn \tcb_doc_head_first:n
  { \tcb_doc_head:nn {#1} {after~ skip=0pt, enlarge~ bottom~ by=0pt} }
\cs_new:Nn \tcb_doc_head_middle:n
  { \tcb_doc_head:nn {#1} {before~ skip=0pt, after~ skip=0pt, enlarge~ bottom~ by=0pt} }
\cs_new:Nn \tcb_doc_head_last:n
  { \tcb_doc_head:nn {#1} {before~ skip=0pt} }
\cs_generate_variant:Nn \tcb_doc_head_single:n { V }
\cs_generate_variant:Nn \tcb_doc_head_first:n { V }
\cs_generate_variant:Nn \tcb_doc_head_last:n { V }

% expand to single {tcb@doc@head} env
% #1 = options for command head
% #2 = options for {tcb@doc@head} env
\cs_new:Nn \tcb_doc_head:nn
  {
    \tcbset{doc/.cd, #1}
    \begin{tcb@doc@head}{doc@head@command, #2}
    % it is better to append \strut inside \tcb@Print@Com
    \strut
    \tcb@Print@Com{\kvtcb@doc@name}
    % write no-duplicate indexes per environment
    \clist_if_in:NVF \l_tcb_doc_cmdnames_clist { \kvtcb@doc@name }
      {
        \tcb@index@Com{\kvtcb@doc@name}
        \clist_gput_right:NV \l_tcb_doc_cmdnames_clist { \kvtcb@doc@name }
      }
    % write no-duplicate labels per environment
    \clist_if_in:NVF \l_tcb_doc_labelnames_clist { \kvtcb@doc@label }
      {
        \protected@edef\@currentlabel{\noexpand\tcb@cs{\kvtcb@doc@name}}
        \label{com:\kvtcb@doc@label}
        \clist_gput_right:NV \l_tcb_doc_labelnames_clist { \kvtcb@doc@label }
      }
    {\ttfamily \kvtcb@doc@parameter}
    \tcb@doc@do@description
    \end{tcb@doc@head}
  }


%%
%% a key-value scheme
%%

% #1 = tcb options
% #2 = default doc-head, clist "name=..., para=..., desp=..., label=..."
% #3 = clist of variant doc-head
% see more discussion in https://github.com/T-F-S/tcolorbox/issues/89
\NewDocumentEnvironment{docCommands-keyvals}{ O{} m O{} }
  {
    \tcbset{#1}
    \begin{tcb@manual@entry}
    \tcb_doc_init_per_env:
    % the only difference compared to {docCommands} env
    \tcb_doc_heads_kv:nn {#2} {#3}
    \nobreak\tcbset{before~ upper=}
    \kvtcb@doc@body@command@before
    \ignorespaces
  }
  {
    \ifvmode\else\unskip\fi
    \kvtcb@doc@body@command@after
    \end{tcb@manual@entry}
  }

\cs_new:Nn \tcb_doc_heads_kv:nn
  {
    % init
    \clist_clear:N \l_tcb_doc_full_heads_clist
    \clist_clear:N \l_tcb_doc_cmdnames_clist
    % parse default doc-head, store results in \l_tcb_doc_full_heads_clist
    \tcb_doc_head_parse_kv:n {#1}
    % parse variant doc-heads, use extra group to achive inheriting
    \clist_map_inline:nn {#2}
      { \group_begin: \tcb_doc_head_parse_kv:n {##1} \group_end: }
    % call \tcb_doc_heads:V
    \tcb_doc_heads:V { \l_tcb_doc_full_heads_clist }
  }

% parse kv list and append results to \l_tcb_doc_full_heads_clist
% #1 = kv list, "name=..., para=..., desc=..."
\cs_new:Nn \tcb_doc_head_parse_kv:n
  {
    \tcbset{doc/.cd, #1}
    % inner "doc label" inherits from "doc name", not the outer "doc label"
    \prop_set_from_keyval:Nn \l_tcb_temp_prop {#1}
    \prop_if_in:NnF \l_tcb_temp_prop {label}
      { \let\kvtcb@doc@label\kvtcb@doc@name }
    % append full doc-head clist (\l_tcb_temp_clist) to \l_tcb_doc_full_heads_clist
    \clist_clear:N \l_tcb_temp_clist
    \clist_map_inline:Nn \c_tcb_doc_head_options_clist
      {
        \meta_clist_put_kv_right:Nnv \l_tcb_temp_clist
          {##1} { kvtcb@doc@ ##1 }
      }
    \meta_clist_brace_gput_right:NV \l_tcb_doc_full_heads_clist
      { \l_tcb_temp_clist }
  }

\cs_new:Nn \meta_clist_put_kv_right:Nnn
  {
    \clist_put_right:Nn #1 { #2 = #3 }
  }
\cs_generate_variant:Nn \meta_clist_put_kv_right:Nnn { Nnv }

% put #2 right to clist #1, with extra pair of braces
\cs_new:Nn \meta_clist_brace_gput_right:Nn
  {
    \clist_gput_right:Nn #1 { {#2} }
  }
\cs_generate_variant:Nn \meta_clist_brace_gput_right:Nn { NV }

\makeatother
\ExplSyntaxOff

\endinput


%% ----------------------------
%% begin: original definitions, v4.22
%%   from tcbdocumentation.code.tex
%%   link https://github.com/T-F-S/tcolorbox/blob/master/tex/latex/tcolorbox/tcbdocumentation.code.tex
%% ----------------------------

% v4.21, new option "doc label"
\newenvironment{docCommand}[3][]{\tcbset{doc label={#2},#1}%
  \begin{tcb@manual@entry}%
  \begin{tcb@doc@head}{doc@head@command}%
  \tcb@Print@Com{#2}\tcb@index@Com{#2}\protected@edef\@currentlabel{\noexpand\tcb@cs{#2}}\label{com:\kvtcb@doc@label}{\ttfamily #3}%
  \tcb@doc@do@description%
  \end{tcb@doc@head}\nobreak\tcbset{before upper=}\kvtcb@doc@body@command@before\ignorespaces}%
  {\ifvmode\else\unskip\fi\kvtcb@doc@body@command@after\end{tcb@manual@entry}}

\newenvironment{tcb@manual@entry}{\begin{list}{}{%
  \setlength{\leftmargin}{\kvtcb@doc@left}%
  \setlength{\itemindent}{0pt}%
  \setlength{\itemsep}{0pt}%
  \setlength{\parsep}{0pt}%
  \setlength{\rightmargin}{\kvtcb@doc@right}%
  }\item}{\end{list}}

\newtcolorbox{tcb@doc@head}[1]{blank,colback=white,colframe=white,
  code={\tcbdimto\tcb@temp@grow@left{-\kvtcb@doc@indentleft}%
        \tcbdimto\tcb@temp@grow@right{-\kvtcb@doc@indentright}},
  grow to left by=\tcb@temp@grow@left,%
  grow to right by=\tcb@temp@grow@right,
  sidebyside,sidebyside align=top,
  sidebyside gap=-\tcb@w@upper@real,
  phantom=\phantomsection,%
  enlarge bottom by=-0.2\baselineskip,#1}

%% ----------------------------
%% end: original definitions
%% ----------------------------
