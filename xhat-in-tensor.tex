% questions on
% - https://wenda.latexstudio.net/q-1115.html and
% - https://www.zhihu.com/question/359661536
\documentclass{article}
\usepackage{amsmath}
\usepackage{bm}
\usepackage{tikz}
\usetikzlibrary{arrows.meta}
\usepackage{xparse}

\makeatletter
\newsavebox\xhat@box
\newsavebox\xhat@box@pre
\newsavebox\xhat@box@suf
\newsavebox\xhat@box@voffset

\tikzset{
  xhat line/.style={
    line cap=round
  },
  xhat dot/.style={
    midway, fill=black, draw=white, 
    circle, 
    line width=1pt, inner sep=1pt
  },
  xhat arrow/.style={
    arrows={-Latex[length=3pt]}
  }
}

\ExplSyntaxOn
\NewDocumentCommand \xhat { O{} m }{
  \xhat:nn {#1} {#2}
}

\tl_new:N \xhat_whole_tl

% #1 = phantom height text
% #2 = whole text
\cs_new:Nn \xhat:nn
  {
    \tl_set:Nn \xhat_whole_tl {#2}
    \xhat@setbox{@pre}    {\tl_head:N  \xhat_whole_tl}
    \xhat@setbox{@suf}    {\tl_item:Nn \xhat_whole_tl {-1}}
    \xhat@setbox{@voffset}{#1}
    \xhat@setbox{}        {\xhat_whole_tl}
    \mathnormal{
      \mathop{
        \mathstrut
        \xhat@voffset{\xhat@box@voffset}
        \xhat@voffset{\xhat@box}
        \smash[t]{\usebox\xhat@box}
      }
      \limits^{
        \begin{tikzpicture}[baseline={(beg)}]
          \path 
            (.5\wd\xhat@box@pre, 0)                coordinate (beg)
            (\wd\xhat@box - .5\wd\xhat@box@suf, 0) coordinate (end);
          \draw[white] (0, 0) -- +(\wd\xhat@box, 0);
          \draw[xhat~ line] (beg) -- node[xhat~ dot] {} (end);
          \path[xhat~ line, xhat~ arrow]
            (beg) edge +(0, -5pt)
            (end) edge +(0, -5pt);
        \end{tikzpicture}
      }
    }
  }

\def\xhat@setbox#1#2{
  \expandafter\savebox\csname xhat@box#1\endcsname {\ensuremath{#2}}
}
\def\xhat@voffset#1{
  \rule{0pt}{\dimexpr\ht#1 - 2pt\relax}
}
\ExplSyntaxOff
\makeatother

\input code-with-output

\begin{document}
\begin{example}{Simple syntax of \Verb|\xhat|}
  Simple syntax: \Verb|\xhat{<text>}|
  \[ xxx \xhat{abcd} yyy = \xhat{ST} \]
  
  Be careful: surround the first and last ``compound symbol" in braces or not?
  \begin{align*}
    abc \xhat{a^2bc}ddd   &= \xhat{ST_2}    \tag{without braces}  \\
    abc \xhat{{a^2}bc}ddd &= \xhat{S{T_2}}  \tag{with braces}     \\
    abc \xhat{a^2bc}ddd   &= \xhat{S{T_2}}  \tag{more preferable}
  \end{align*}
\end{example}

\begin{example}{Full syntax of \Verb|\xhat|}
  Full syntax: \Verb|\xhat[<phantom height text>]{<text>}|
  \begin{align*}
    \xhat{{\bm{T}}{\bm{S}}} &=
      \xhat{{\bm{g}}^{k^{2^2}}\bm{g}^l\bm{g}_r{\bm{g}_s}}
                              \tag{different heights} \\
    \xhat[\bm{g}^{k^{2^2}}]{{\bm{T}}{\bm{S}}} &=
      \xhat{{\bm{g}}^{k^{2^2}}\bm{g}^l\bm{g}_r{\bm{g}_s}}
                                    \tag{same height}
  \end{align*}
\end{example}
\end{document}
