\documentclass{article}
\usepackage{amsmath}
\usepackage{amssymb}
\usepackage{nicematrix}
\usetikzlibrary{calc, matrix, shapes.misc}


\input code-with-output
\begin{document}

\begin{example}[sidebyside]{Plain \TeX{} \texttt{\textbackslash bordermatrix}}
  \[
    \bordermatrix{
        & i & j & k \cr
      1 & a & b & c \cr
      2 & d & e & f
    }
  \]
\end{example}


\begin{example}[sidebyside, righthand width=3cm]{\LaTeXe{} \texttt{array} environment}
  \[
    \left[\begin{array}{lcr}
    	a_1     &   b_1   &     c_1 \\
    	a_{11}  & b_{11}  &  c_{11} \\
    	a_{111} & b_{111} & c_{111}
    \end{array}\right]
  \]
\end{example}


\makeatletter
\newlength\@delta   \setlength\@delta{5pt}
\newlength\@yOffset \setlength\@yOffset{5pt}
\newlength\@xOffset \setlength\@xOffset{5pt}

% re-define to adjust position of matrix delimiter
\def\tikz@delimiter#1#2#3#4#5#6#7#8{%
  \bgroup
    \pgfextra{\let\tikz@save@last@fig@name=\tikz@last@fig@name}%
    node[outer sep=0pt, inner sep=0pt, 
      draw=none, fill=none, anchor=#1, 
      at=(\tikz@last@fig@name.#2), #3, yshift=\@yOffset]
    {%
      {\nullfont\pgf@process{%
        \pgfpointdiff
          {\pgfpointanchor{\tikz@last@fig@name}{#4}}
          {\pgfpointanchor{\tikz@last@fig@name}{#5}}%
      }}%
      $\left#6
        \vcenter{\hrule height \dimexpr .5#8 - \@delta\relax 
                        depth  \dimexpr .5#8 - \@delta\relax 
                        width 0pt}
        \hskip-\@xOffset\relax\right#7$%
    }
    \pgfextra{\global\let\tikz@last@fig@name=\tikz@save@last@fig@name}%
  \egroup
}
\makeatother

\begin{example}{\texttt{Ti\textit{k}Z matrix}, annotations on matrix}
  % \usepackage{amssymb}
  % \usepackage{tikz}
  % \usetikzlibrary{matrix, shapes.misc}
  \tikzset{
    m common/.style={draw, inner sep=1pt},
    mo/.style={m common, circle},
    mx/.style={m common, cross out}
  }
  
  \begin{tikzpicture}[baseline=(magic-2-1.base)]
    \matrix (magic) [matrix of nodes, 
      left delimiter={[}, right delimiter={]}, 
      above delimiter=\{, below delimiter=\}] {
      8 & |[mo]|1 & |[mx]|6 \\
      3 &       5 &       7 \\
      4 &       9 &       2 \\
    };
    
    \draw[thick, blue] (magic-3-1.west) -- (magic-3-3.east);
    \node[xshift=10pt] at (magic-2-3.east) {$\checkmark$};
  \end{tikzpicture}
\end{example}


\ExplSyntaxOn
\makeatletter

\newcommand\phantomHeight[1]{\rule[#1]{0pt}{0pt}}
\newcommand\phantomDepth[1]{\rule[-#1]{0pt}{0pt}}

\newcommand\drawAboveDelimiter{\drawDelimiter{north}}
\newcommand\drawBelowDelimiter{\drawDelimiter{south}}

% Draw extented delimiter #2 which spreads from node (#3.west) to (#4.east).
%   This macro imitates the definition of \tikz@delimiter, the macro behind 
%   options of matrix library like "above delimiter" and "below delimiter".
% #1 = name of node anchor
% #2 = delimiter symbol, e.g., \rbrace
% #3 = name of beginning node
% #4 = name of ending node
% #5 = extra tikz options act on the delimiter
\def\drawDelimiter#1#2#3#4#5{
  \begin{tikzpicture}
    \node[outer~ sep=0pt, inner~ sep=0pt, 
      draw=none, fill=none, #5, rotate=-90] 
      at ($ (#3.#1)!.5!(#4.#1) $)
    {
      {\nullfont\pgf@process{
        \pgfpointdiff
          {\pgfpointanchor{nm - \int_use:N\g__nm_env_int - #3 -large}{west}}
          {\pgfpointanchor{nm - \int_use:N\g__nm_env_int - #4 -large}{east}}
      }}
      $\left.\vcenter{\nullfont\hrule height .5\pgf@x depth .5\pgf@x width0pt}\right#2$
    };
  \end{tikzpicture}
}

\makeatother
\ExplSyntaxOff

\begin{example}{\texttt{nicematrix}, add above and below delimiters}
  % \usepackage{nicematrix}
  % \usetikzlibrary{calc}
  \[
    \begin{pNiceArray}{CCC}[last-col, first-row, create-extra-nodes, 
        code-for-first-row=\phantomDepth{10pt},
        code-after={
          \drawAboveDelimiter{\lbrace}{1-1}{1-3}{yshift=.7em, blue}
          \drawBelowDelimiter{\rbrace}{3-1}{3-3}{yshift=-.7em, teal}
        }]
      0 & 0 & 0 &     \\  % \\[10pt] does not work, 
      1 & 2 & 3 & L_1 \\  % see `texdoc nicematrix`, sec. 6
      4 & 5 & 6 & L_2 \\
      7 & 8 & 9 & L_3
    \end{pNiceArray}
  \]
\end{example}

\end{document}